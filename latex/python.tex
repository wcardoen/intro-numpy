\documentclass[11pt]{article}
\usepackage[english]{babel} % Language to be used
\usepackage{color}
\usepackage{url}
\usepackage{listings}
\usepackage[pdftex]{hyperref}
%\usepackage{hyperref}
\hypersetup{
  pdftitle={Introduction to Numpy},
  pdfauthor={Wim R.M. Cardoen, PhD},
  pdfpagelayout=SinglePage,
  bookmarks=true,
  citecolor=green,
  colorlinks=true,
  linkcolor=red,
  urlcolor=blue
}

\begin{document}
\title{Introduction to Numpy}
\author{Wim R.M. Cardoen \& Brett Milash\\
        University of Utah\\
        Center for High Performance Computing\\
        155 S. 1452 E., Rm 405\\
        Salt Lake City\\
        UT, 84112\\
        USA\\
        \texttt{wim.cardoen@utah.edu}\\
	\texttt{brett.milash@utah.edu}}
\renewcommand{\today}{July 21, 2020}
\renewcommand{\labelitemii}{$\star$}
\maketitle

\section*{Introduction}
In this document we describe the installation of the tutorial/lecture notes 
and/or verification of the required software.

\renewcommand \thesection{\Roman{section}}
\section{Installation of the tutorial}
The tutorial/lecture notes can be retrieved as follows:
\begin{itemize}
\item either by using Git\footnote{In this case we presume that the Git software is installed on your machine. 
	The Git software can be freely obtained from the following site: \href{https://git-scm.com/}{https://git-scm.com/}}:\newline
		\href{git clone https://github.com/wcardoen/numpy-pearc20.git}{git clone https://github.com/wcardoen/numpy-pearc20.git}
\item by downloading the following zip file:\newline
	\href{https://github.com/wcardoen/numpy-pearc20/archive/master.zip}{https://github.com/wcardoen/numpy-pearc20/archive/master.zip}
\end{itemize}
In the former case the local directory \texttt{numpy-pearc20} 
will be created. 

In what follows we assume that the newly created directory bears the name \texttt{numpy-pearc20}.  


\section{Required software}
\begin{itemize}
  \item Python $>=3.6$ \newline
        We will be using Python $>=3.6$ during the workshop.
        We also require the following Python packages to be installed:
        \begin{itemize}
           \item numpy $>=$ 1.12.0 
           \item scipy $>=$ 0.19.0
           \item matplotlib $>=$ 1.5.0
           \item notebook $>=$ 4.2.0
        \end{itemize} 

  \item Jupyter \newline
        The lectures will be presented using Jupyter notebooks.
\end{itemize}

In the tutorial materials there is a script \texttt{check\_env.py} 
(see Section \ref{section:check}) that can be used to check your installation.


\section{Check your Python installation}\label{section:check}
The packages that are required for this tutorial are frequently available 
on most OSs.
Therefore, it is quite likely that you don't have to install any additional packages.

After installing the tutorial, you can immediately check whether the required packages 
are present on your system. \newline\newline
\texttt{
\$ cd numpy-pearc20\footnote{The lines starting with a \$ sign are commands to be executed in a Shell.} \newline
\$ python3 check\_env.py\footnote{The output generated by this command will be dependent on the installed version of Python and the required packages.}\newline
}

\begin{verbatim}
sleipnir@ragnarok:~/Talks/numpy-pearc20$ python3 check_env.py


  Checking installation ...

    Python 3.7.6 found
        --prefix=/home/sleipnir/software/pkg/anaconda3/2020.02

            numpy  inst.:    1.18.1  >= req.:    1.12.0 -> OK 
            scipy  inst.:     1.4.1  >= req.:    0.19.0 -> OK 
       matplotlib  inst.:     3.1.3  >= req.:     1.5.0 -> OK 
         notebook  inst.:     6.0.3  >= req.:     4.2.0 -> OK 

    All required packages are installed
    Congratulations!

  End checking installation
\end{verbatim}  


You can verify the presence of Jupyter like this:\newline\newline
\texttt{
\$ jupyter --version}\footnote{The output generated will depend on the Jupyter version installed on your machine.}
\begin{verbatim}
sleipnir@ragnarok:~/Talks/numpy-pearc20$ jupyter --version
jupyter core     : 4.6.1
jupyter-notebook : 6.0.3
qtconsole        : 4.6.0
ipython          : 7.12.0
ipykernel        : 5.1.4
jupyter client   : 5.3.4
jupyter lab      : 1.2.6
nbconvert        : 5.6.1
ipywidgets       : 7.5.1
nbformat         : 5.0.4
traitlets        : 4.3.3
\end{verbatim}

If some of the above packages are missing you can proceed with 
the installation step (section\,\ref{section:install}). Otherwise 
you are completely ready for the tutorial.
 
\section{Installation of the required software}\label{section:install}
If you don't have Python$3$ installed or you have an incomplete installation 
(see section\,\ref{section:check}),
installing the \href{https://www.anaconda.com/download/}{Anaconda distribution} is 
the easiest option.

The Anaconda distribution is available for the Mac, Windows \& Linux OSs. 
It also contains all the aforementioned requirements/packages.

Return to section\,\ref{section:check} to verify that your installation is properly working.

%\bibliographystyle{alpha}
%\bibliography{ref}
\end{document}
