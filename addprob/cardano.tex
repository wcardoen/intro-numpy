\documentclass[10pt]{article}
\usepackage{amsmath}
\usepackage{amssymb}
\usepackage{amsfonts}
\usepackage{amsthm}
\usepackage[english]{babel} % Language to be used
\usepackage{color}
\usepackage{url}
\usepackage{listings}
\usepackage{hyperref}
\hypersetup{
  pdftitle={Extra exercises},
  pdfauthor={Wim R.M. Cardoen, PhD},
  pdfpagelayout=SinglePage,
  bookmarks=true,
  citecolor=green,
  colorlinks=true,
  linkcolor=red,
  urlcolor=blue
}

\urlstyle{same}

% Parameters to the margins
\setlength{\topmargin}{-0.50in}
\setlength{\oddsidemargin}{-0.25in}
\setlength{\evensidemargin}{-0.25in}
\setlength{\textwidth}{7.0in}
\setlength{\textheight}{9.00in}

\begin{document}
\title{Extra exercises}
\author{Wim R.M. Cardoen \& Brett Milash\\
        University of Utah\\
        Center for High Performance Computing\\
        155 S. 1452 E., Rm 405\\
        Salt Lake City\\
        UT, 84112\\
        USA\\
        \texttt{wim.cardoen@utah.edu}\\
	\texttt{brett.milash@utah.edu}}
\renewcommand{\today}{February 18, 2024}
\renewcommand{\labelitemii}{$\star$}
\maketitle

\begin{enumerate}
	\item \texttt{Cardano triplets}\footnote{\href{https://www.projecteuler.net/problem=251}{Project Euler - Problem $251$ (slimmed down version)}}\newline
  	 The triplet of integers $(a,b,c)$ where $a,b,c \in \mathbb{N}_0$ bears the name 
         of a \texttt{Cardano triplet} if it satisfies the following condition:
  	 \begin{equation}
  	    \displaystyle \sqrt[3]{a+b\sqrt{c}}\,+\, \sqrt[3]{a-b\sqrt{c}}   =   1 \nonumber
	 \end{equation}		
         \begin{itemize}
            \item e.g. $(2,1,5)$ is a \texttt{Cardano triplet}
            \item there exists $149$ \texttt{Cardano triplets} for which $a\,+\,b\,+\,c\,\le\,1000$
            \item how many \texttt{Cardano triplets} exist for which $a\,+\,b\,+\,c\,\le\,10000000$?
         \end{itemize}			 
\end{enumerate}
\renewcommand \thesection{\Roman{section}}

\end{document}
